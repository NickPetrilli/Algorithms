\documentclass{article}
\usepackage[utf8]{inputenc}
\author{Nicholas Petrilli }


\title{	
   \normalfont \normalsize 
   \textsc{CMPT 435 - Fall 2021 - Dr. Labouseur} \\[10pt] % Header stuff.
   \textsc{Assignment Two}
}
\date{\today}

\begin{document}

\maketitle

\section{Results}
\begin{center}
\begin{tabular}{||c c c||} 
 \hline
  & Number of Comparisons & Asymptotic Running Time  \\ [0.5ex] 
 \hline\hline
 Selection Sort & 220780 & $O(n^2)$ \\ 
 \hline
 Insertion Sort & 114121 & $O(n^2)$\\
 \hline
 Merge Sort & 5413 & O(n*log(n)) \\
 \hline
 Quick Sort & 1836 & O(n*log(n)) \\ [1ex]
 \hline
\end{tabular}
\end{center}

The asymptotic running times for both selection sort and insertion sort is $O(n^2)$. Both of these functions iterate through nested for loops which cause the Big Oh upper bound to be $n^2$. On the other hand, merge sort and quick sort have asymptotic running times of O(n*log(n)). Instead of using for loops, these algorithms use recursion and break up the sorting into two steps: divide and conquer. The running time for dividing the array is log(n), and the running time for conquering the array until its sorted is n which makes up the n*log(n). Also, the shuffle method is taking in a completely sorted array and it is randomly moving the elements around, so the number of comparisons is slightly less than expected. 


\end{document}
