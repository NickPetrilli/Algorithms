\documentclass{article}
\usepackage[utf8]{inputenc}
\usepackage{listings}

\title{Algorithms Semester Project}
\author{Nicholas Petrilli}


\title{	
   \normalfont \normalsize 
   \textsc{CMPT 435 - Fall 2021 - Dr. Labouseur} \\[10pt] % Header stuff.
   \textsc{Semester Project}
}



\begin{document}


\lstset{
  numbers=left,
  firstnumber=1,
  numberfirstline=true
}
\maketitle


\section{Results}
\begin{center}
\begin{tabular}{||c c||} 
 \hline
  & Number of Tests Administered  \\ [0.5ex] 
 \hline\hline
 1,000 People & 255 \\ 
 \hline
 10,000 People & 2,560 \\
 \hline
 100,000 People & 25,686 \\
 \hline
 1,000,000 People & 256,206 \\ [1ex]
 \hline
\end{tabular}
\end{center}

\noindent Listed in the table above is an average of 10 runs for each population size. For a disease with an infection rate of 2 percent these are values that we expect when running the simulation. The slight difference in results when running the simulation for larger population sizes is a result of the distribution method used in this simulation. This simulation models binomial distribution, or selection without replacement which results in the same probability for each trial. When randomly choosing which people in the population get infected, it would be better to remove the infected people to eliminate the possibility of them being infected again. This is the subtle difference between binomial distribution and hypergeometric distribution. Hypergeometric distrubution uses selection with replacement, so when one person gets infected they are already accounted for and the probability slightly increases for the others in the population to be infected. In other words, in binomial distribution the events are indepedent of eachother, whereas in hypergeometric distribution the probability is dependent on the previous draw. This difference in distributions would make the simulation more like real life, which is the overall goal. Another way to improve the simulation would be to account for the fact that the tests aren't perfect and may result in false positives, or even worse false negatives. Also instead of hardcoding the infection rate and group size, letting the user input both of those values will give this simulation more use for different diseases.



\section{PooledTesting Class}

\begin{lstlisting} [language = Java, caption = Pooled Testing with Main Method]

import java.util.*;

public class PooledTesting {
    
    public static int infectionRate = 2; // represents 2% infection rate
    public static int groupSize = 8;
    public static int numTests = 0;
    

    public static void main(String[] args) {

        System.out.println("Welcome to my Pooled Testing Simulation! In this simulation the disease has an infection rate of 2% and individuals will be tested in groups of 8.");
        System.out.println("Enter a population size that you want to test on. (1,000, 10,000, 100,000 or 1,000,0000)");
        System.out.print("Population size: ");

        Scanner input = new Scanner(System.in);
        int populationSize = input.nextInt();

        List<Person> peopleList = new ArrayList<>(populationSize);
        ListPeople listPeople = new ListPeople(peopleList);

        listPeople.addPeople(populationSize);
        input.close();

        System.out.println("\n--Running testing simulation for " + populationSize + " people--\n");

        infect(peopleList, listPeople);

        test(peopleList, listPeople);

        System.out.println("\nNumber of tests needed for " + populationSize + " people is " + numTests + "\n");

   
        


    

    }
    public static void infect(List<Person> peopleList, ListPeople listPeople) {
        //infect population with disease with 2% infection rate
        listPeople.giveDisease(infectionRate);
        int infectionCount = 0;
        for (int i = 0; i < listPeople.size(); i++) {
            if (peopleList.get(i).getIsSick() == 1) {
                infectionCount++;
                //printing out for greater than 10,000 is too many 
                if (peopleList.size() <= 10000) {
                    System.out.println("Person " + i + " has been infected");
                }//if

            }//if
        }//for
        System.out.println("The total number of people infected for population size " + peopleList.size() + " is " + infectionCount);
    }

    public static void test(List<Person> peopleList, ListPeople listPeople) {
        /*
        testing group size at a time (8) so split up list into groups of 8 
        if infection is found
            split into two lists
            if one group shows infection and other does not
                everyone in infect group tested individually, other group clear
            else both groups show infection
                test all members of both groups
        */

    
        List<List<Person>> listOfLists = splitInGroups(listPeople.getList(), groupSize);
        //iterate through each list of 8 
        for (List<Person> list: listOfLists) {
            numTests++;
            //then iterate through each person in list
            for (Person person: list) {
                //test if anyone of the 8 are sick
                if (person.getIsSick() == 1) {
                    //need to split this list into two groups of 4 here
                    List<List<Person>> splitList = new ArrayList<>();
                    splitList = splitInTwo(list);
                    //iterate through new list of lists with the new groups of 4
                    //increment test
                    for (int i = 0; i < splitList.size(); i++) {
                        numTests++;
                    }
                    //now need to iterate through the two lists of 4
                    for (List<Person> splitGroup: splitList) {
                        //iterate through the individual people in the lists of 4
                        for (Person personInSplitGroup: splitGroup) {
                            //if anyone is sick, need to test every person in each of the two groups of 4
                            if (personInSplitGroup.getIsSick() == 1) {
                                for (int j = 0; j < splitGroup.size(); j++) {
                                    numTests++;
                                }//for
                            }//if
                        }//for
                    }//for

                }//if
            }//for
        }//for
    }

    //splits original list up into groups of 8
    //returns a list of lists (of 8 each)
    public static <T> List<List<T>> splitInGroups(List<T> list, int groupSize) {
        List<List<T>> listOfLists = new ArrayList<List<T>>();
        for (int i = 0; i < list.size(); i += groupSize) {
            //adds sublist of original list from i to groupSize
            //i + groupSize exceeds the list size for the last list, so the min function is used for that case
            listOfLists.add(new ArrayList<T>(list.subList(i, Math.min(list.size(), i + groupSize))));
        }
        return listOfLists;
    }

    //split list of 8 into two lists of 4
    //also returns list of list (the two lists of 4)
    public static <T> List<List<T>> splitInTwo(List<T> list) {
        List<List<T>> listsOfFour = new ArrayList<List<T>>();

        int size = list.size();
    
        List<T> first = new ArrayList<>(list.subList(0, (size + 1) / 2));
        List<T> second = new ArrayList<>(list.subList((size + 1) / 2, size));

        listsOfFour.add(first);
        listsOfFour.add(second);

        return listsOfFour;


    }
    public static void resetTests() {
        numTests = 0;
    }


\end{lstlisting}


\section{ListPeople Class}

\begin{lstlisting} [language = Java, caption = ListPeople Class]

import java.util.*;

public class ListPeople {
    
    private List<Person> listPeople = new ArrayList<>();
    
    //List of people constructor
    public ListPeople(List<Person> listPeople) {
        this.listPeople = listPeople;
    }

    //Adds specified number of people into list to represent population
    public void addPeople(int numPeople) {
        for (int i = 0; i < numPeople; i++) {
            Person person = new Person(0);
            listPeople.add(person);
        }
    }

    //Infection rate is set to 2%, so this infects 2% of the population
    //Generates random number between 0 and 100, if its less than 2 the person gets infected
    public void giveDisease(int infectionRate) {
        for (int i = 0; i < listPeople.size(); i++) {
            Random rand = new Random();
            int percentSick = rand.nextInt(100);
            if (percentSick < infectionRate) {
                listPeople.get(i).setIsSick(1);
            }
        }
    }

    public int size() {
        return listPeople.size();
    }

    public List<Person> getList() {
        return listPeople;
    }
    
}


\end{lstlisting}

\section{Person Class}

\begin{lstlisting} [language = Java, caption = Person Class]

public class Person {

    //0 represents not sick, 1 reprents sick
    private int isSick = 0;

    //construct person object
    public Person(int isSick) {
        this.isSick = isSick;
    }
    
    public void setIsSick(int isSick) {
        this.isSick = isSick;
    }
    public int getIsSick() {
        return this.isSick;
    }
    
}

\end{lstlisting}




\end{document}
